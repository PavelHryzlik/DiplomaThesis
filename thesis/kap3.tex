\chapter{Otevřené smlouvy jako Linked Data}
\label{sec:kap4}

V minulé kapitole [\ref{sec:kap3}] bylo řečeno, jak reprezentovat smlouvy jako otevřená data. Data serializovaná do formátu JSON, či CSV můžeme kategorizovat stupněm 3$\bigstar$. Pokud však chceme dosáhnout otevřenosti dat kategorie 5$\bigstar$, je třeba provést několik dalších kroků:

\begin{itemize}
\item Identifikovat reprezentované objekty a vlastnosti pomocí URI
\item Vytvořit strojově srozumitelné struktury, resp. napojit data na konkrétní slovníky tříd a predikátů - ontologie
\item Propojit smlouvy pomocí odkazů na související data
\end{itemize}

\section{Přiřazení identifikátorů jednotlivým entitám otevřených smluv}

K jednoznačné identifikaci každé entity nám stačí její \textit{typ} a \textit{Id}. Výjimku tvoří dokumenty, které jsou verzované. K nim je nutné přidat informaci o konkrétní verzi. Dále chceme vyjádřit vztah podřízené entity k nadřízené. Řešením je opět přidání informací o typu podřízené entity, příp. jejího \textit{Id}. Resp. základní URI schéma bude:

\bigskip

\textit{http://[domain]/[typ]/[entitaId]/{[versionId]}/{[childEntity]}/{[childEntityId]}}
\begin{itemize}
\item domain je doména instituce publikující smlouvy
\item údaje ve složených závorkách jsou volitelné
\end{itemize}

\medskip
\noindent
Výsledné identifikátory entit:

\begin{itemize}
\item \textbf{Document} (Dokument) - \textit{http://[domain]/[type]/[documentId]/[versionId]}
	\begin{itemize}
	\item Type může nabývat hodnot contract/attachment/amendment - resp. jedná se v podstatě o hodnotu položky Uri z datového standardu 
	\end{itemize}
\item \textbf{Publisher} (Vydavatel) -\textit{http://[domain]/publisher}
	\begin{itemize}
	\item Jedná se o podřízenou položku dokumentu. Předpokládáme, že instituce publikující na konkrétní doméně odpovídá konkrétnímu vydavateli smluv. Je vhodné, aby každý vydavatel publikoval pod unikátní doménou. 
	\end{itemize}
\item \textbf{Version} (Verze) - \textit{http://[domain][type]/[documentId]/[versionId]/version}
	\begin{itemize}
	\item Jedná se o podřízenou položku dokumentu 
	\end{itemize}
\item \textbf{Order} (Objednávka) - \textit{http://[domain]/order/[orderId]}
\item \textbf{Invoice} (Faktura) - \textit{http://[domain]/invoice/[invoiceId]}
\item \textbf{Implementation} (Implementace)
	\begin{itemize}
	\item \textit{http://[domain]/[type]/[documentId]/[versionId]/implementation}
		\begin{itemize}
		\item Pokud se jedná o podřízenou položku dokumentu
		\end{itemize}
	\item \textit{http://[domain]/order/[orderId]/implementation}
		\begin{itemize}
		\item Pokud se jedná o  podřízenou položku objednávky
		\end{itemize}
	\end{itemize}
\item \textbf{Milestone} (Milník) 
\begin{itemize}
	\item Zde si dovolíme drobné zjednodušení, a to vynechání implementation z identifikátoru. Adresa bude jednodušší, informační hodnota však zůstane stejná
	\item \textit{http://[domain]/[type]/[id]/[versionId]/milestone/[milestoneId]}
		\begin{itemize}
		\item Milník u dokumentu
		\end{itemize}
	\item \textit{http://[domain]/order/[orderId]/milestone/[milestoneId]}
		\begin{itemize}
		\item Milník u objednávky
		\end{itemize}
	\end{itemize}
\item \textbf{Transaction} (Transakce)
\begin{itemize}
	\item Zjednodušení, viz Milník
	\item \textit{http://[domain]/[type]/[id]/[versionId]/transaction/[transactionId]}
		\begin{itemize}
		\item Transakce u dokumentu
		\end{itemize}
	\item \textit{http://[domain]/order/[orderId]/transaction/[transactionId]}
		\begin{itemize}
		\item Transakce u objednávky
		\end{itemize}
	\end{itemize}
\item \textbf{Party} (Smluvní strana) - \textit{http://[domain]/party/[partyId]}
\item \textbf{Address} (Adresa) - \textit{http://[domain]/party/[partyId]/address}
	\begin{itemize}
	\item Jedná se o podřízenou položku smluvní strany 
	\end{itemize}
\item \textbf{SuperiorInstitution} (Nadřazená instituce) - \\\textit{http://[domain]/superiorInstitution/[superiorInstitutionId]}
\end{itemize}

\section{Ontologie pro publikaci dat o smlouvách}

Než začneme s tvorbou ontologie, je dobré si uvědomit, že vycházíme z již hotového datového standardu. Nemáme tedy v tvorbě úplnou svobodu. Cílem tedy bude tvorba takové ontologie, která bude odpovídat stávajícímu datovému standardu, a přesto se bude snažit využít co nejvíce již existujících ontologií.

\bigskip
\noindent
Samotnou tvorbu ontologie rozdělíme do dvou kroků:

\begin{itemize}
\item Analýza vhodných, již existujících ontologií
\item Tvorba samotné ontologie
\end{itemize}

\subsection{Analýza vhodných, již existujících ontologií}

Při tvorbě ontologie se zaměříme na otázku, zdali existuje třída, či predikát v nějaké ontologii sémanticky ekvivalentní třídě, či konkrétní položce datového standardu pro smlouvy. Takových vhodných ontologií ve světě Linked Data může být celá řada. K výběru stačí libovolná z nich. 

V následujícím seznamu je výčet vybraných ontologií, které se jeví jako vhodné pro použití při popisování smluv\footnote{Obecně výběr ontologií nemusíme považovat za striktní. Každou třídu, či predikát lze označit jako sémanticky ekvivalentní jiné třídě, či predikátu. Slouží k tomu konstrukce jazyka Owl - \textit{equivalentClass}, resp. \textit{equivalentProperty}.}. U každého bodu je zmíněn popis ontologie, důvod, proč byla daná ontologie zvolena, a seznam tříd a predikátů vhodných k použití.

\bigskip
\noindent
Vybrané ontologie:

\begin{itemize}
\item \textbf{Commerce} (\textit{https://w3id.org/commerce\#}) - ontologie pro popisování obchodních transakcí
	\begin{itemize}
	\item Důvod použití - užitečná třída transakce 
	\item Vybrané třídy 
		\begin{itemize}
		\item Transaction - třída reprezentující transakci 
		\end{itemize}
	\item Vybrané predikáty
		\begin{itemize}
		\item contentUrl - adresa obsahu
		\item source - zdroj transakce (pozor na party)
		\item destination - cíl transakce  (pozor na party)
		\end{itemize}
	\end{itemize}
\item \textbf{DublinCore} (\textit{http://purl.org/dc/terms/}) - základní ontologie pro popis metadat 
	\begin{itemize}
	\item Důvod použití - základní a všeobecně známá ontologie popisující metadata
	\item Vybrané predikáty
		\begin{itemize}
		\item created - datum vytvoření
		\item creator - tvůrce
		\item date - obecné datum
		\item description - popis metadat
		\item identifier - jednoznačný identifikátor
		\item issued - datum publikace
		\item language - jazyk
		\item modified - datum modifikace
		\item publisher - vydavatel
		\item rights - licence
		\item title - název dokumentu
		\item type - typ dokumentu
		\end{itemize}
	\end{itemize}
\item \textbf{Friend-of-a-Friend} (\textit{http://xmlns.com/foaf/0.1}) - ontologie pro popis vazeb mezi lidmi
	\begin{itemize}
	\item Důvod použití - vhodná pro označení třídy vydavatele
	\item Vybrané třídy 
		\begin{itemize}
		\item Person - třída reprezentující osobu
		\item Organization - třída reprezentující organizaci 
		\end{itemize}
	\item Vybrané predikáty
		\begin{itemize}
		\item name - jméno osoby
		\item mbox - email osoby
		\end{itemize}
	\end{itemize}
\item \textbf{GoodRelations} (\textit{http://purl.org/goodrelations/v1\#}) - ontologie pro popis produktů, cen a obchodních dat
	\begin{itemize}
	\item Důvod použití - známá ontologie, vhodná pro popis smluvních stran a informací o cenách 
	\item Vybrané třídy 
		\begin{itemize}
		\item BusinessEntity - třída popisující hospodářské subjekty
		\end{itemize}
	\item Vybrané predikáty
		\begin{itemize}
		\item hasCurrency - měna
		\item hasCurrencyValue - cena
		\item legalName - název subjektu
		\item valueAddedTaxIncluded - plátce DPH
		\end{itemize}
	\end{itemize}
\item \textbf{PaySwarm} (\textit{https://w3id.org/payswarm\#}) - ontologie popisující účtenky, platby, pronájmy a obecně výměnné platby na webu
	\begin{itemize}
	\item Důvod použití - obsahuje predikáty popisující intervaly platnosti
	\item Vybrané predikáty
		\begin{itemize}
		\item validFrom - datum platnosti od
		\item validUntil - datum platnosti do
		\end{itemize}
	\end{itemize}
\item \textbf{Schema} (\textit{http://schema.org/}) - obecná ontologie mající za cíl pokrývat co největší možné množství informací
	\begin{itemize}
	\item Důvod použití - známá ontologie, možnost využití pro popis adresních údajů 
	\item Vybrané třídy 
		\begin{itemize}
		\item PostalAddress - třída reprezentující adresu 
		\end{itemize}
	\item Vybrané predikáty
		\begin{itemize}
		\item address - adresa
		\item addressCountry - země
		\item addressLocality - město
		\item postalCode - PSČ
		\item streetAddres - ulice
		\end{itemize}
	\end{itemize}
\item \textbf{Vann} (\textit{http://purl.org/vocab/vann/}) - anotační ontologie pro dokumenty 
	\begin{itemize}
	\item Důvod použití - nesouvisí s datových standardem, tato ontologie je vhodná pro popis ontologií a bude zmíněna níže v publikaci [\ref{sec:kap3}].
	\item Vybrané predikáty
		\begin{itemize}
		\item preferredNamespaceUri - preferovaná adresa ontologie
		\item preferredNamespacePrefix - preferovaná zkratka
		\item usageNote - poznámka k použití
		\end{itemize}
	\end{itemize}

\end{itemize}

\subsection{Tvorba ontologie}

Každou položku datového standardu namapujeme na třídu, či predikát výsledné ontologie. Pro některé položky využijeme zmíněné predikáty z již existujících ontologií, pro ostatní vytvoříme třídy a predikáty vlastní. 

Vlastní ontologii nazveme jako open-contracting a budeme pro ni používat zkratku cn.  

Výsledné mapování můžeme vidět v následujících tabulkách 4.1 - 4.13. V prvním sloupečku se nachází entita/vlastnost datového standardu, v druhém napamovaná třída/predikát a v třetím případná poznámka\footnote{Entity jsou uváděny velkým písmem, vlastnosti malým}.

\subsubsection*{Dokument}

\begin{center}
\begin{longtable}{lll}
\label{grid_mlmmh} \\
\multicolumn{1}{l}{\textbf{Entita/Vlastnost}} & 
\multicolumn{1}{l}{\textbf{Třída/Predikát}} & 
\multicolumn{1}{l}{\textbf{Poznámka}} \\ \hline 
\endfirsthead
\multicolumn{1}{l}{\textbf{Entita/Vlastnost}} & 
\multicolumn{1}{l}{\textbf{Třída/Predikát}} & 
\multicolumn{1}{l}{\textbf{Poznámka}} \\ \hline 
\hline
\endhead
\endfoot
\caption{Mapování entity Document}
\endlastfoot
\textbf{uri} & cn:uri \\
\textbf{document} & com:contentUrl \\
\textbf{versions} & cn:version \\
\textbf{type} & dcmi:type \\
\textbf{publisher} & dc:publisher \\
\textbf{valid} & cn:valid \\
\textbf{plainText} & cn: plainText \\
\textbf{responsiblePersons} & cn:responsiblePerson \\
\textbf{anonymised} & cn:anonymised \\
\end{longtable}
\end{center}

\newpage

\subsubsection*{Vydavatel}

\begin{center}
\begin{longtable}{lll}
\label{grid_mlmmh} \\
\multicolumn{1}{l}{\textbf{Entita/Vlastnost}} & 
\multicolumn{1}{l}{\textbf{Třída/Predikát}} & 
\multicolumn{1}{l}{\textbf{Poznámka}} \\ \hline 
\endfirsthead
\multicolumn{1}{l}{\textbf{Entita/Vlastnost}} & 
\multicolumn{1}{l}{\textbf{Třída/Predikát}} & 
\multicolumn{1}{l}{\textbf{Poznámka}} \\ \hline 
\hline
\endhead
\endfoot
\caption{Mapování entity Vydavatel}
\endlastfoot
\textbf{Publisher} & foaf:Organization \\
\textbf{id} & dc:identifier \\
\textbf{name} & gr:legalName \\
\textbf{noID} & cn:noID \\
\textbf{country} & schema:addressCountry \\
\textbf{authentication} & cn:authentication \\
\end{longtable}
\end{center}

\subsubsection*{Verze}

\begin{center}
\begin{longtable}{lll}
\label{grid_mlmmh} \\
\multicolumn{1}{l}{\textbf{Entita/Vlastnost}} & 
\multicolumn{1}{l}{\textbf{Třída/Predikát}} & 
\multicolumn{1}{l}{\textbf{Poznámka}} \\ \hline 
\endfirsthead
\multicolumn{1}{l}{\textbf{Entita/Vlastnost}} & 
\multicolumn{1}{l}{\textbf{Třída/Predikát}} & 
\multicolumn{1}{l}{\textbf{Poznámka}} \\ \hline 
\hline
\endhead
\endfoot
\caption{Mapování entity Verze smlouvy}
\endlastfoot
\textbf{Version} & cn:Version \\
\textbf{publisherId} & cn:publisherId \\
\textbf{version} & cn:versionOrder \\
\textbf{uri} & cn:uri \\
\textbf{published} & dc:issued \\
\end{longtable}
\end{center}

\subsubsection*{Smlouva}

\begin{center}
\begin{longtable}{lll}
\label{grid_mlmmh} \\
\multicolumn{1}{l}{\textbf{Entita/Vlastnost}} & 
\multicolumn{1}{l}{\textbf{Třída/Predikát}} & 
\multicolumn{1}{l}{\textbf{Poznámka}} \\ \hline 
\endfirsthead
\multicolumn{1}{l}{\textbf{Entita/Vlastnost}} & 
\multicolumn{1}{l}{\textbf{Třída/Predikát}} & 
\multicolumn{1}{l}{\textbf{Poznámka}} \\ \hline 
\hline
\endhead
\endfoot
\caption{Mapování entity Smlouva TODO-PriceSpec}
\endlastfoot
\textbf{Contrac}t & cn:Contract \\
\textbf{awardID} & cn:awardID \\
\textbf{awardProfileID} & cn:awardProfileID \\
\textbf{amount} & gr:hasCurrencyValue \\
\textbf{amountNoVat} & gr:hasCurrencyValue \\
\textbf{title} & dc:title \\
\textbf{contractType} & cn:contractType \\
\textbf{parties} & cn:party \\
\textbf{subjectType} & cn:subjectType \\
\textbf{priceAnnual} & cn:priceAnnual \\
\textbf{currency} & gr:hasCurrency \\
\textbf{dateSigned} & dc:created \\
\textbf{validFrom} & ps:validFrom \\
\textbf{validUntil} & ps:validUntil \\
\textbf{funding} & cn:funding \\
\textbf{attachments} & cn:attachment \\
\textbf{amendments} & cn:amendment \\
\textbf{competency} & cn:competency \\
\textbf{currentValidContract} & cn:currentValidContract \\
\textbf{description} & dc:description \\
\textbf{implementation} & cn:implementation \\
\end{longtable}
\end{center}

\subsubsection*{Příloha}

\begin{center}
\begin{longtable}{lll}
\label{grid_mlmmh} \\
\multicolumn{1}{l}{\textbf{Entita/Vlastnost}} & 
\multicolumn{1}{l}{\textbf{Třída/Predikát}} & 
\multicolumn{1}{l}{\textbf{Poznámka}} \\ \hline 
\endfirsthead
\multicolumn{1}{l}{\textbf{Entita/Vlastnost}} & 
\multicolumn{1}{l}{\textbf{Třída/Predikát}} & 
\multicolumn{1}{l}{\textbf{Poznámka}} \\ \hline 
\hline
\endhead
\endfoot
\caption{Mapování entity Příloha}
\endlastfoot
\textbf{Attachment} & cn:Attachment \\
\textbf{title} & dc:title \\
\textbf{contract} & cn:contract \\
\textbf{number} & cn:attachmentOrder \\
\end{longtable}
\end{center}

\subsubsection*{Dodatek}

\begin{center}
\begin{longtable}{lll}
\label{grid_mlmmh} \\
\multicolumn{1}{l}{\textbf{Entita/Vlastnost}} & 
\multicolumn{1}{l}{\textbf{Třída/Predikát}} & 
\multicolumn{1}{l}{\textbf{Poznámka}} \\ \hline 
\endfirsthead
\multicolumn{1}{l}{\textbf{Entita/Vlastnost}} & 
\multicolumn{1}{l}{\textbf{Třída/Predikát}} & 
\multicolumn{1}{l}{\textbf{Poznámka}} \\ \hline 
\hline
\endhead
\endfoot
\caption{Mapování entity Dodatek}
\endlastfoot
\textbf{Amendment} & cn:Amendment \\
\textbf{title} & dc:title \\
\textbf{contract} & cn:contract \\
\textbf{number} & cn:amendmentOrder \\
\textbf{dateSigned} & dc:created \\
\end{longtable}
\end{center}

\subsubsection*{Smluvní strana}

\begin{center}
\begin{longtable}{lll}
\label{grid_mlmmh} \\
\multicolumn{1}{l}{\textbf{Entita/Vlastnost}} & 
\multicolumn{1}{l}{\textbf{Třída/Predikát}} & 
\multicolumn{1}{l}{\textbf{Poznámka}} \\ \hline 
\endfirsthead
\multicolumn{1}{l}{\textbf{Entita/Vlastnost}} & 
\multicolumn{1}{l}{\textbf{Třída/Predikát}} & 
\multicolumn{1}{l}{\textbf{Poznámka}} \\ \hline 
\hline
\endhead
\endfoot
\caption{Mapování entity Smluvní strana}
\endlastfoot
\textbf{Party} & gr:BusinessEntity \\
\textbf{id} & dc:identifier \\
\textbf{localID} & cn:localID \\
\textbf{name} & gr:legalName \\
\textbf{payer} & cn:payer \\
\textbf{noID} & cn:noID \\
\textbf{country} & schema:addressCountry \\
\textbf{paysVAT} & gr:valueAddedTaxIncluded \\
\textbf{superiorInstitution} & cn:superiorInstitution \\
\end{longtable}
\end{center}

\subsubsection*{Nadřazená instituce}

\begin{center}
\begin{longtable}{lll}
\label{grid_mlmmh} \\
\multicolumn{1}{l}{\textbf{Entita/Vlastnost}} & 
\multicolumn{1}{l}{\textbf{Třída/Predikát}} & 
\multicolumn{1}{l}{\textbf{Poznámka}} \\ \hline 
\endfirsthead
\multicolumn{1}{l}{\textbf{Entita/Vlastnost}} & 
\multicolumn{1}{l}{\textbf{Třída/Predikát}} & 
\multicolumn{1}{l}{\textbf{Poznámka}} \\ \hline 
\hline
\endhead
\endfoot
\caption{Mapování entity Nadřazená instituce}
\endlastfoot
\textbf{SuperiorInstitution} & gr:BusinessEntity \\
\textbf{id} & dc:identifier \\
\textbf{localID} & cn:localID \\
\textbf{name} & gr:legalName \\
\textbf{payer} & cn:payer \\
\textbf{noID} & cn:noID \\
\textbf{country} & schema:addressCountry \\
\end{longtable}
\end{center}

\subsubsection*{Adresa}

\begin{center}
\begin{longtable}{lll}
\label{grid_mlmmh} \\
\multicolumn{1}{l}{\textbf{Entita/Vlastnost}} & 
\multicolumn{1}{l}{\textbf{Třída/Predikát}} & 
\multicolumn{1}{l}{\textbf{Poznámka}} \\ \hline 
\endfirsthead
\multicolumn{1}{l}{\textbf{Entita/Vlastnost}} & 
\multicolumn{1}{l}{\textbf{Třída/Predikát}} & 
\multicolumn{1}{l}{\textbf{Poznámka}} \\ \hline 
\hline
\endhead
\endfoot
\caption{Mapování entity Nadřazená instituce}
\endlastfoot
\textbf{Address} & schema:PostalAddress \\
\textbf{streetAddress} & schema:streetAddres \\
\textbf{locality} & schema:addressLocality \\
\textbf{postalCode} & schema:postalCode \\
\textbf{nuts} & cn:nuts \\
\end{longtable}
\end{center}

\subsubsection*{Objednávka}

\begin{center}
\begin{longtable}{lll}
\label{grid_mlmmh} \\
\multicolumn{1}{l}{\textbf{Entita/Vlastnost}} & 
\multicolumn{1}{l}{\textbf{Třída/Predikát}} & 
\multicolumn{1}{l}{\textbf{Poznámka}} \\ \hline 
\endfirsthead
\multicolumn{1}{l}{\textbf{Entita/Vlastnost}} & 
\multicolumn{1}{l}{\textbf{Třída/Predikát}} & 
\multicolumn{1}{l}{\textbf{Poznámka}} \\ \hline 
\hline
\endhead
\endfoot
\caption{Mapování entity Objednávka}
\endlastfoot
\textbf{Order} & cn:Order \\
\textbf{parrentDocument} & cn:parrentDocument \\
\textbf{subjectType} & cn:subjectType \\
\textbf{parties} & cn:party \\
\textbf{title} & dc:title \\
\textbf{amount} & gr:hasCurrencyValue \\
\textbf{currency} & gr:hasCurrency \\
\textbf{dateSigned} & dc:created \\
\textbf{implementation} & cn:implementation \\
\end{longtable}
\end{center}

\subsubsection*{Faktura}

\begin{center}
\begin{longtable}{lll}
\label{grid_mlmmh} \\
\multicolumn{1}{l}{\textbf{Entita/Vlastnost}} & 
\multicolumn{1}{l}{\textbf{Třída/Predikát}} & 
\multicolumn{1}{l}{\textbf{Poznámka}} \\ \hline 
\endfirsthead
\multicolumn{1}{l}{\textbf{Entita/Vlastnost}} & 
\multicolumn{1}{l}{\textbf{Třída/Predikát}} & 
\multicolumn{1}{l}{\textbf{Poznámka}} \\ \hline 
\hline
\endhead
\endfoot
\caption{Mapování entity Faktura}
\endlastfoot
\textbf{Invoice} & cn:Invoice \\
\textbf{parrentDocument} & cn:parrentDocument \\
\textbf{parties} & cn:party \\
\textbf{title} & dc:title \\
\textbf{amount} & gr:hasCurrencyValue \\
\textbf{currency} & gr:hasCurrency \\
\textbf{dateSigned} & dc:created \\
\textbf{dueDate} & cn:dueDate \\
\end{longtable}
\end{center}

\subsubsection*{Rozšiřující entity}

\subsubsection*{Implementace}

\begin{center}
\begin{longtable}{lll}
\label{grid_mlmmh} \\
\multicolumn{1}{l}{\textbf{Entita/Vlastnost}} & 
\multicolumn{1}{l}{\textbf{Třída/Predikát}} & 
\multicolumn{1}{l}{\textbf{Poznámka}} \\ \hline 
\endfirsthead
\multicolumn{1}{l}{\textbf{Entita/Vlastnost}} & 
\multicolumn{1}{l}{\textbf{Třída/Predikát}} & 
\multicolumn{1}{l}{\textbf{Poznámka}} \\ \hline 
\hline
\endhead
\endfoot
\caption{Mapování entity Implementace}
\endlastfoot
\textbf{Implementation} & cn:Implementation \\
\textbf{milestones} & cn:milestone \\
\textbf{transactions} & cn:transaction \\
\end{longtable}
\end{center}

\subsubsection*{Milník}

\begin{center}
\begin{longtable}{lll}
\label{grid_mlmmh} \\
\multicolumn{1}{l}{\textbf{Entita/Vlastnost}} & 
\multicolumn{1}{l}{\textbf{Třída/Predikát}} & 
\multicolumn{1}{l}{\textbf{Poznámka}} \\ \hline 
\endfirsthead
\multicolumn{1}{l}{\textbf{Entita/Vlastnost}} & 
\multicolumn{1}{l}{\textbf{Třída/Predikát}} & 
\multicolumn{1}{l}{\textbf{Poznámka}} \\ \hline 
\hline
\endhead
\endfoot
\caption{Mapování entity Milník}
\endlastfoot
\textbf{Milestone} & cn:Milestone \\
\textbf{title} & dc:title \\
\textbf{dueDate} & cn:dueDate \\
\end{longtable}
\end{center}

\subsubsection*{Transakce}

\begin{center}
\begin{longtable}{lll}
\label{grid_mlmmh} \\
\multicolumn{1}{l}{\textbf{Entita/Vlastnost}} & 
\multicolumn{1}{l}{\textbf{Třída/Predikát}} & 
\multicolumn{1}{l}{\textbf{Poznámka}} \\ \hline 
\endfirsthead
\multicolumn{1}{l}{\textbf{Entita/Vlastnost}} & 
\multicolumn{1}{l}{\textbf{Třída/Predikát}} & 
\multicolumn{1}{l}{\textbf{Poznámka}} \\ \hline 
\hline
\endhead
\endfoot
\caption{Mapování entity Transakce}
\endlastfoot
\textbf{Transaction} & com:Transaction \\
\textbf{date} & dc:date \\
\textbf{amount} & gr:hasCurrencyValue \\
\textbf{senderOrganization} & com:source \\
\textbf{receiverOrganization} & com:destination \\ 
\textbf{publisherId} & cn:publisherId \\
\end{longtable}
\end{center}

\subsection{Publikace}
\label{sec:kap423}

K serializaci výsledné ontologie využijeme formátu Turtle. Soubor se skládá z hlavičky a definicí nově vytvořených tříd a predikátů. V hlavičce definujeme prefixy použitých ontologií a základní informace o ontologii. Použité třídy a predikáty zmíníme v poznámkách k použití (predikát \textit{vann:usageNote}). Příklad ontologie lze nalézt v Příloze D.

\section{Možnosti propojení na související data}

První bezpochyby zajímavou možností je propojení smluv s veřejnými zakázkami, resp. věstníkem veřejných zakázek provozovaným Ministerstvem pro místní rozvoj. Jednotlivé smlouvy mající spojitost s veřejnou zakázkou poznáme podle vlastnosti AwardID, resp. evidenční číslo veřejné zakázky.  

Dalšími zajímavými prvky k propojení jsou pokročilé informacemi o smluvním stranách. Každá smluvní strana vystupující ve smlouvě může mít zveřejněno mimo jiné identifikační číslo a adresu. Nabízí se tedy propojení s národními registry ARES a RÚIAN. ARES je registrem informací o ekonomických subjektech provozovaný Ministerstvem financí, RÚIAN je registrem územní identifikace, adres a nemovitostí provozovaný Českým úřadem zeměměřickým a katastrálním. Vhodné informace k propojení tedy jsou:

\begin{itemize}
\item Evidenční číslo veřejné zakázky u smlouvy k propojením s věstníkem veřejných zakázek
	\begin{itemize}
	\item Iniciativa OpenData.cz zpracovává údaje o veřejných zakázkách v RDF podobě, využijeme tedy propojení právě s tímto datasetem 
	\item Cílové URL -\\http://linked.opendata.cz/resource/domain/buyer-profiles/contract/cz/{EvidencniCisloZakazky}
	\end{itemize}
\item Identifikační číslo smluvní strany s možností propojení na ARES
	\begin{itemize}
	\item OpenData.cz aktuálně zpracovávají také údaje z registru ARES
	\item Cílové URL - http://linked.opendata.cz/resource/business-entity/CZ{ICO}\footnote{Informace skrývající se za tímto odkazem jsou sjednocením více datových zdrojů, ne pouze ARESu} 
	\end{itemize}
\item Adresa smluvní strany na RÚIAN - TODO
\end{itemize}

\section{Provázání s datovým formátem JSON}

V třetí kapitole jsme si ukázali, jak publikovat smlouvy v datovém formátu JSON. Vzhledem k budoucímu možnému využití v aplikacích pracujících nad smlouvami v JSON dokumentech by bylo dobré nastínit, jak taková data rozšířit, aby se z nich stala zároveň RDF data. Cílem je ale zachovat původní strukturu JSON souboru, resp. aby data byla validní vůči JSON schématu. Pro tyto účely je ideální formát JSON-LD. Jediné, co nám stačí, je v původních datech každému objektu přiřadit \textit{@id}, \textit{@type} a definovat \textit{@context} (viz první kapitola Příklad serializovaných dat ve formátu JSON-LD). Při tvorbě ontologie jsme si popsali mapování entit a položek z datového standardu na konkrétní třídy a predikáty. V kontextu je tedy přesně takovéto mapování, viz příklad kódu \ref{lst:contract_context}. Na příkladu kódu \ref{lst:contract_example_without_context} je již vidět výsledný JSON-LD soubor s jednou smlouvou a dvěma smluvními stranami (pro porovnání, původní JSON soubor viz kód \ref{lst:contract_example}). Jedná se tedy o RDF data, která popisuje námi definovaná ontologie a zároveň jde o data splňující datový standard, resp. jsou validní vůči JSON Schématu. Hlavním přínosem je to, že RDF data serializovaná v takto definovaném JSON-LD formátu budou použitelná v budoucích aplikacích, příp. registru pracujícím nad datovým standardem.

\lstinputlisting[label=lst:contract_context, caption=JSON-LD Context, language=XML]{code/contract_context.jsonld}

\lstinputlisting[label=lst:contract_example_without_context, caption=JSON-LD Soubor s jednou smlouvou, language=XML]{code/contract_example_without_context.jsonld}