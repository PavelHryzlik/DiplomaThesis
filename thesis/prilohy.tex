\chapwithtoc{A Příloha}

\section*{Harmonogram událostí v souvislosti s otevřenými smlouvami}

\begin{table}[h]
\begin{tabular}{lp{120mm}}
\textbf{2.12.2013} & Předložen návrh zákona o Registru smluv \\
\textbf{26.11.2014} & Seminář Transparentnost v obcích - Myšlenka datového standardu pro otevřené smlouvy \\
\textbf{4.12.2014} & Schůzka akční skupiny k tvorbě datového standardu pro otevřené smlouvy na radnici Praha 6 \\
\textbf{6.1.2015} & Schůzka akční skupiny k potvrzení datového standardu pro otevřeného smlouvy na radnici Praha 6 \\
\textbf{6.2.2015} & Představen spolek Otevřená města \\
\textbf{18.3.2015} & Schůzka s Jiřím Skuhrovcem k projektu Metodika zveřejňování smluv \\
\textbf{29.8.2015} & Odeslán datový standard pro otevřené smlouvy Ministerstvu vnitra (ve formě CSV) \\
\textbf{18.9.2015} & Schválen zákon o Registru smluv poslaneckou sněmovnou  \\
\textbf{21.10.2015} & Zakládající konference spolku Otevřená města  \\
\textbf{22.10.2015} & Zákon vrácen senátem s pozměňovacími návrhy  \\
\textbf{24.11.2015} & Sněmovna setrvala na původním návrhu zákona \\
\end{tabular}
\end{table}

\chapwithtoc{B Příloha}

\section*{Uživatelská dokumentace}

Adresa konverzního mechanismu je nastavena na:
\begin{itemize}
	\item \textit{http://[domain]/sparql}
\end{itemize}
Základní funkcionalita konverzního modulu a webové aplikace je k nalezení v kapitole Implementace platformy. U obou projektů stačí k běžnému nastavení soubor \textit{Web.config}. Projekty byly vyvíjeny ve vývojovém prostředí Visual Studia 2013, později ve verzi 2015. Základní předpoklady pro využití platformy jsou:

\begin{itemize}
\item Konverzní mechanismus
	\begin{itemize}
	\item Prostředí, kde lze publikovat webovou aplikace v prostředí .NET, tedy Windows, Windows Server, IIS, MS Azure apod.
	\item MSSQL Server 2012 a vyšší pro funkci R2RML procesoru
	\item Přístup k MSSQL databázi firmy Triada, spol. s r. o. 
	\item Přístup k datovému úložišti firmy Triada, spol. s r. o.
	\item Knihovna pro práci s datovým úložištěm 
	\item R2RML mapovací skript
	\end{itemize}
\item Jednotné úložiště
	\begin{itemize}
		\item Nástroj UnifiedViews
		\item Triplestore databáze, např. Openlink Virtuoso Universal Server
		\item Konfigurační soubor s datovým katalogem požadovaných datasetů
	\end{itemize}
\item Webová aplikace
	\begin{itemize}
	\item Prostředí, kde lze publikovat webovou aplikace v prostředí .NET, tedy Windows, Windows Server, IIS, MS Azure apod.
	\item Přístup k SPARQL endpointům:
		\begin{itemize}
		\item \textit{http://student.opendata.cz/sparql}
		\item \textit{http://linked.opendata.cz/sparql}
		\item \textit{http://ruian.linked.opendata.cz/sparql}
		\item \textit{http://cs.dbpedia.org/sparql}
		\end{itemize}	
	\end{itemize}
\end{itemize}



\chapwithtoc{C Příloha}

\section*{Obsah přiloženého datového nosiče}

Základní struktura přiloženého datového nosiče je rozdělena do těchto složek (zmíníme také klíčové skripty): 

\begin{table}[h]
\begin{tabular}{lp{55mm}}
\textbf{/ContractStandard/} & Složka datového standardu \\
\textbf{/ContractStandard/lod/} & Složka se skripty využívající principy LinkedData \\
\textbf{/ContractStandard/lod/contract\_context.jsonld} & JSON-LD Context - mapující skript \\
\textbf{/ContractStandard/lod/contract\_ontology.ttl} & Ontologie pro otevřené smlouvy \\
\textbf{/ContractStandard/lod/subject\_catalog.ttl} & Datový katalog pro UnifiedViews \\
\textbf{/ContractStandard/lod/triada\_esmluv\_r2rml.ttl} & R2RML mapovací skript \\
\textbf{/ContractStandard/samples/} & Složka s příklady otevřených smluv \\
\textbf{/ContractStandard/schema/} & Složka se schématy datového standardu \\
\textbf{/ContractStandard/schema/contract\_schema.csv} & Datový standard pro otevřené smlouvy - CSV \\
\textbf{/ContractStandard/schema/contract\_schema.json} & Datový standard pro otevřené smlouvy - JSON Schema \\
\textbf{/ContractViewer/} & Složka s projektem webové aplikace  \\
\textbf{/TestResults/} & Složka s testy konverzního modulu a web. aplikace \\
\textbf{/TriadaEndpoint/} & Složka s projektem konverzního modulu \\
\textbf{/UnifiedViews/} & Složka se skripty použité v rámci pipeline \\
\textbf{/UnifiedViews/UnifiedViewsExport/} & Vyexportované soubory popisující pipeline nástroje UnifiedViews \\
\textbf{/thesis.pdf} & Text práce \\
\end{tabular}
\end{table}

\section*{Online zdroje}

Projekty řešené v rámci této práce lze nalézt na těchto na adresách:

\begin{itemize}
\item[] \textit{https://github.com/PavelHryzlik/DiplomaThesis} - Text práce
\item[] \textit{https://github.com/PavelHryzlik/ContractStandard} - Datový standard
\item[] \textit{https://github.com/PavelHryzlik/TriadaEndpoint} - Konverzní modul
\item[] \textit{https://github.com/PavelHryzlik/ContractViewer} - Webová aplikace
\item[] \textit{http://opencontracts.azurewebsites.net/} - Testovací nasazení web. aplikace
\end{itemize}

\chapwithtoc{D Příloha}

\lstinputlisting[basicstyle=\scriptsize, label=lst:contract_ontology, title=RDF Ontologie pro smlouvy, language=Xml]{code/contract_ontology.ttl}

\chapwithtoc{E Příloha}

\lstinputlisting[basicstyle=\scriptsize, label=lst:triada_esmluv_r2rml, title=R2RML skript mapující smlouvy, language=Xml]{code/triada_esmluv_r2rml.ttl}
