\chapter{Úvod}

V době informační společnosti se využívání internetu stalo naší každodenní rutinou. Skrze různé webové aplikace a služby každodenně pracujeme s obrovským množstvím informací. Běžně komunikujeme přes e-mail, finance spravujeme skrze internetové bankovnictví, část svého osobního života sdílíme na sociálních sítích. Požadavek na on-line vyřizování agendy vůči veřejné správě tedy není překvapujícím.

Problematika elektronizace veřejné správy, jednotně nazývaná jako \uv{e-govern- ment}, je aktuálním tématem již po mnoho let. Důsledkem tohoto procesu je generování obrovského množství nesmírně důležitých dat. Tato data ale v naprosté většině případů leží schovaná v databázích jednotlivých veřejných institucí. Mnoho z těchto dat by ale ze zákona mělo být volně dostupných. Často však jediným možným způsobem, jak taková data získat je použití zákona č.106/1999 Sb.\cite{z106}, o svobodném přístupu k informacím. Netřeba zmiňovat, že tato snaha se mnohdy může stát značně netriviální.

Řešením je vhodná data, resp. metadata o těchto datech, zpřístupnit on-line. Pro strojově čitelná data zveřejněná na internetu se zažil pojem Otevřená data. Tato data pak může vyhledávat a zpracovávat kdokoli. To přináší řadu dílčích výhod od úspory nákladů, přes boj s korupcí, až po zapojení občanů, nemluvě o podnikatelském potenciálu, převážně možnosti vzniku mnoha užitečných aplikací pracujících nad otevřenými daty. To celé za cenu minimálních nákladů z veřejných rozpočtů.

Otevírání dat můžeme chápat jako další krok v procesu elektronizace veřejné správy. Průkopníky v této oblasti jsou státy s vyspělou formou demokracie, jako USA a Spojené království. Příklad si ale také můžeme vzít od Estonska. Malá země, vědoma si, že nemá nerostné bohatství ani rozvinutý průmysl, se rozhodla prosadit na poli informačních technologií, kde základem jsou otevřené on-line služby veřejné správy. Důležitost otevřených dat si uvědomuje i Evropská unie. Směrnicí 2013/37/EU\cite{smeu} v podstatě doporučuje členským státům, aby data otevíraly. 
České republice se také povedlo nastartovat procesy otevírání veřejné správy. Pokrok je cítit hlavně na národní úrovni. Mezi městy a obcemi jsou však otevřená data často stále neznámým pojmem. Problematikou a obecně osvětou otevřených dat se zabývá mimo jiné Ministerstvo vnitra ČR\cite{mv}, projekt Rekonstrukce státu\cite{rek}, Fond Otakara Motejla\cite{fom}, Oživení o.s.\cite{oz}, fórum pro otevřená data\cite{otevrenadata}, či iniciativa OpenData.cz\cite{od}.

Otevřená data však nelze chápat jako samospásné řešení problémů veřejné správy. Jsou spíše prostředkem ke zvýšení otevřenosti a transparentnosti. Veřejná služba však může být netransparentní i s otevřenými daty. Řekněme, že pro kvalitní veřejnou službu jsou otevřená data nutnou, nikoli však postačující podmínkou.

Dalším aspektem otevřených dat je jejich kvalita. Kvalitní otevřená data jsou propojena mezi sebou v rámci jednotného sdíleného prostoru, mohou na sebe odkazovat a využívat širokého kontextu, které takový sdílený prostor propojených dat nabízí. Taková data využívají principů Linked Data.\cite{opendatapsi, opendatagovernment, opendatacr, odgov_s}

\section{Motivace}

Základní motivaci pro vznik této práce bych rozdělil do tří pilířů:

\subsection*{Veřejnoprávní sféra}

Na podzim roku 2014 se konal seminář Transparentnost v obcích\cite{spt} v Poslanecké sněmovně pořádaný panem Mgr. Janem Farským. V rámci semináře se sešla skupina složená ze zástupců měst a obcí, akademické sféry a neziskového sektoru. Předmětem jednání byla otevřená data. Výsledkem bylo rozhodnutí, že první datovou sadou vhodnou k plošnému otevření, také vzhledem k chystanému zákonu o registru smluv, jsou údaje o smlouvách. Prvním krokem je standardizace datového formátu, resp. určení položek vhodných ke zveřejnění. Motivací bylo, že pokud standard začne využívat netriviální počet měst a obcí, tak je reálná šance k prosazení standardu na národní úroveň. Ustanovila se tedy, pod zášťitou Oživeni o.s. a EconLabu (dříve Centra aplikované ekonomie o.s.)\cite{econLab}, \uv{akční} skupina, jejímž cílem byla tvorba datového standardu pro otevřené smlouvy. Bylo mi ctí stát se členem této skupiny.

\subsection*{Komerční sféra}

Jako externista se podílím na tvorbě software pro veřejnou správu ve společnosti Triada spol. s.r.o. Mým úkolem se ke konci roku 2014 stala tvorba modulu ESML pro interní evidování smluv.

\subsection*{Akademická sféra}

V rámci MFF UK ve spolupráci s Fakultou informatiky VŠE vznikla iniciativa OpenData.cz. Jejím cílem je vybudování otevřené datové infrastruktury v České republice. Na MFF UK také probíhá výzkum propojitelných dat, Linked Data. Mým cílem bylo přispět k otevřené datové infrastruktuře, navíc s využitím principů Linked Data. Rozhodnutí věnovat se publikaci dat o smlouvách padlo již v červnu 2014. Konkrétní obrysy však práce získala až s přispěním výše zmíněných pilířů.

Výsledkem je tedy aplikace principů Linked Data pro publikaci a sdílení dat o smlouvách s možností konkrétního využití nad modulem ESML společnosti Triada. To celé s ohledem na vznikající datový standard. Jednou z dílčích motivací bylo, že v případě prosazení datového standardu na národní úroveň mohou města a obce používající modul ESML využitím této práce automaticky zveřejňovat smlouvy v Linked Data podobě, a to s minimálními náklady. Taková data lze pak agregovat do jednotných úložišť, nad kterými mohou vznikat nejrůznější aplikace přinášející konečný přínos pro uživatele.

\section{Cíl práce}

Cílem práce je prozkoumat možnosti využití principů Linked Data pro publikaci a sdílení dat o smlouvách veřejných institucí a jejich propojení na související data ve veřejném prostoru. Prvním krokem je definování datového standardu a ontologie pro otevřené smlouvy. Dalším krokem je návrh způsobu konverze dat stávajícími informačními systémy veřejných institucí (v podobě relačních databází) do otevřeného formátu využívající principy Linked Data a implementace konverzního mechanizmu pro vybraný konkrétní informační systém (Triada spol. s.r.o). V dalším kroku následuje návrh a implementace jednotného úložiště dat o smlouvách v Linked Data s experimentálním zprovozněním na serveru poskytnutém vedoucím práce. V jednotném úložišti se očekává návrh řešení integračních problémů dané heterogenitou dat publikovaných různými veřejnými institucemi. Následujícím krokem je nad tímto jednotným úložištěm návrh a implementace webové aplikace, která data o smlouvách zpřístupní koncovým uživatelům.

\section{Struktura práce}

Obsah práce je rozdělen na 10 kapitol a 5 příloh. Ve druhé kapitole jsou popsány a vysvětleny základní principy otevřených dat. Třetí kapitola se zabývá pojmem otevřené smlouvy. Kapitola nejdříve rozebere aktuální stav otevřenosti smluv ve veřejné správě a následně nastíní vznikající datový standard. Čtvrtá kapitola zadefinuje otevřené smlouvy jako Linked Data. V páté kapitole se analyzují požadavky na platformu pro otevřené smlouvy. Šestá kapitola zmíněnou platformu navrhne. Sedmá kapitola se zabývá konkrétní implementací platformy. V osmé kapitole jsou znázorněny zátěžové testy některých dílčích částí implementace. Devátá kapitola nastíní proces otevírání smluv formou obecné metodiky. Poslední, desátou kapitolou je závěr shrnující práci jako celek. Nedílnou součástí práce je seznam použité literatury, obrázků, tabulek, výpisů kódů a použitých zkratek. Práce zahrnuje také 3 přílohy. V příloze A je znázorněn harmonogram vývoje standardu otevřených smluv. V příloze B se nachází uživatelská dokumentace. Příloha C popisuje strukturu přiloženého datového nosiče. V příloze D se nachází Linked Data ontologie pro smlouvy. Konečně, v příloze E je R2RML skript mapující tabulky z relační databáze do RDF.

